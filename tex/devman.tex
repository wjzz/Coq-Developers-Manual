\documentclass[a4paper,oneside]{book}

%% Simple macros

\newcommand{\m}[1]{\ensuremath{\mathsf{#1}}}
\newcommand{\hint}{\textbf{Hint}:\ }
\newcommand{\TODO}[1]{\textbf{TODO}:\ \emph{#1}}
%% Simple listings

\usepackage{fancyvrb}
\DefineVerbatimEnvironment{code}{Verbatim}{}

%% Package for nice listings

\usepackage{listings}

\lstset{frame=tb,
  aboveskip=3mm,
  belowskip=3mm,
  showstringspaces=false,
  columns=flexible,
  basicstyle={\small\ttfamily},
  numbers=none,
  numberstyle=\tiny\color{gray},
  keywordstyle=\color{blue},
  commentstyle=\color{dkgreen},
  stringstyle=\color{mauve},
  breaklines=true,
  breakatwhitespace=true
  tabsize=3
}

%% Clickable links and urls

\usepackage{url}
\usepackage[colorlinks]{hyperref}

%% Title page info

\author{Wojciech Jedynak}
\date{Wroc{\l}aw, \today}
\title{Coq Developer's Manual}

%% Document start

\begin{document}

\maketitle

\tableofcontents

\part{Introduction}

\chapter{Purpose of the document}

The goal of this manual is to write down and collect in a single place
the main information that a person that (wants to) actively develop
the Coq Proof Assistant should know. We hope that this document will
make it easier for new developers to join the project and dive in. It
would also be great if this manual could provide a place to share
information between the developers without the inconveniences of an
inline comment.

\chapter{Compilation}

\section{Requirements}

To compile Coq from sources you need to have OCaml and the basic OCaml
toolset installed. 

\section{Configuration and first compilation}

To compile the tools for the first time run the following (in the main
Coq directory):

\begin{lstlisting}
$ ./configure -local
$ make
\end{lstlisting}

The first command will check if you have installed all of the required
tools and dependencies. The \verb|-local| option will keep the source
files upon installation, so that we can e.g. compile Coq plugins and
develop Coq in general.

In case of installation problems, check the \verb|INSTALL| file.

\section{Recomplilation}

Once you've been able to compile Coq from sources, you can tweak the
code and then 
\begin{lstlisting}
$ make
\end{lstlisting} will recompile the necessary files. Note
that this command will proceed to recompile the standard library once
the basic executables (e.g., \m{coqc}) are ready. \hint In many cases you
don't really need to wait for the whole process to finish before you 
test the new version.

\chapter{Basic forms of documentation}

\section{Ocamldoc documentation}

Coq consists of dozens of files and modules. By issuing the
\begin{lstlisting}
$ make mli-doc
\end{lstlisting}
command, we can generate a hyperlinked module, type and value
index. The files are created in the \m{dev/ocamldoc/html} directory.
\TODO{We should include some of the indexes at the end of this
  document. We could also inline the comments from all .mli files, on
  a file-by-file basic.}

\chapter{OCaml IDE configuration}

\section{Merlin}

In this section we show how to 
\begin{itemize}
\item install merlin using OPAM
\item configure the Coq project for using merlin inside Emacs
\item take advantage of the nice new features
\end{itemize}

The instructions have been tested under Ubuntu 13.10, OPAM 1.1.1,
OCaml 4.01, Emacs 24.3.1. 

\subsection{Installing merlin}

To install merlin using OPAM run the following commands:
\begin{lstlisting}
opam install merlin                        # if prompted, reply 'Y'
ocamlmerlin -version                       # test if installed OK
\end{lstlisting}

\noindent
Next, configure Emacs by adding the following to your \m{.emacs} file:
\begin{lstlisting}
(setq opam-share (substring (shell-command-to-string "opam config var share") 0 -1))
(add-to-list 'load-path (concat opam-share "/emacs/site-lisp"))
(require 'merlin)
(autoload 'merlin-mode "merlin" "Merlin mode" t)
(add-hook 'tuareg-mode-hook 'merlin-mode)
(add-hook 'caml-mode-hook 'merlin-mode)
\end{lstlisting}

Alternatively you could install merlin from sources, for this visit
merlin's
homepage.\footnote{\url{https://github.com/the-lambda-church/merlin}}

\subsection{Project configuration}

Now we need to configure our project to work with merlin.  First, we
need to create a \m{.merlin} file in Coq's toplevel directory. I
adapted the appropriate file from the trunk version of
Coq.\footnote{\url{https://raw.github.com/coq/coq/trunk/.merlin}}

The \m{.merlin} file for Coq 8.4pl3:
\begin{lstlisting}
FLG -rectypes

S config
B config
S lib
B lib
S kernel
B kernel
S kernel/byterun
B kernel/byterun
S library
B library
S pretyping
B pretyping
S interp
B interp
S proofs
B proofs
S tactics
B tactics
S parsing
B parsing
S toplevel
B toplevel
S plugins
B plugins

PKG lablgtk2.sourceview2
S ide
B ide
S ide/utils
B ide/utils
S tools
B tools
S tools/coqdoc
B tools/coqdoc
S dev
B dev

S checker
B checker
\end{lstlisting}

(S is for listing source code directories, B is for binary (\m{.cmi}
and \m{.cmt}) files.) To check if this project file is put in the
right place, open one of the \m{.ml} files in Emacs and run
\verb|M-x merlin-goto-project-file|.

Next we need to make sure we compile OCaml files with the
-\m{bin}-\m{annot} compiler flag, which tells the compiler to emit
\m{.cmt} files, which contain additional information for the
IDEs. Find the following lines in the \m{Makefile.build} configuration
file:

\begin{lstlisting}
OCAMLC += $(CAMLFLAGS)
OCAMLOPT += $(CAMLFLAGS)
\end{lstlisting}

and change the file so that part reads

\begin{lstlisting}
CAMLFLAGS += -bin-annot         # add this line
OCAMLC += $(CAMLFLAGS)
OCAMLOPT += $(CAMLFLAGS)
\end{lstlisting}

Now, recompile all of the source files (by \verb|make clean; make|).

\subsection{Features}

\textbf{Note}: Merlin only works on \m{.ml} files!

\noindent
Merlin integrates rather nicely with e.g. Tuareg and it provides you
with the following features:

\begin{itemize}
\item Type-checking on save: whenever you save the current ML buffer
  (by \verb|C-x C-s|) the file is type-checked. If everything is OK,
  then you get an 'ok' message in the mini-buffer. Otherwise, the
  offending lines/points become highlighted and you can navigate
  through the error points by \verb|C-c C-x| and the mini-buffer will
  contain the error message.
\item Finding the definition of a function/identifier: \verb|C-c C-l|
  will locate the symbol at point and go to its
  definition. \verb|C-c &| will then go back to the original point.
\item Type information: \verb|C-c C-t| on an identifier will print its
  type in the mini-buffer. \verb|C-c t| will prompt of an expression
  and also print its type. What is really neat, is that this command
  will use the current context (at point)!
\item Auto completion: this comes in two flavors: \verb|M-<tab>| and
  \verb|C-c <tab>|. The second version requires the
  \m{auto}-\m{complete} Emacs package and is more convenient: the
  first version opens a list of candidate expansions in a separate
  buffer.
\end{itemize}

To update Merlin's state run 'update' \verb|C-c C-u| or 'restart'
\verb|C-c r|.

Most of the behaviors described above can be configured or disabled,
for details consult Merlin's github
page.\footnote{\url{https://github.com/the-lambda-church/merlin/wiki/emacs-from-scratch}}

\part{The code base: structure, concepts and conventions}

\chapter{General remarks and concepts}

In this chapter we provide a purpose-oriented tour of the source
files. We begin by listing some of the core modules, which are
imported by virtually all other modules.

\section{Reading order for source files}

Some files depend on as many as 20+ modules, while others are very
often opened. 

In the \m{kernel} directory, \m{names.ml} and \m{term.ml} seem to be a
good starting point.

The \m{lib/util.ml} file is used to collect any code that can be often
used, so it's useful to have a general idea of what can be found there.

Besides \m{kernel/term.ml}, \m{interp/topconstr.ml} and
\m{pretyping/glob\_term.ml} will be often consulted, because they
define the abstract syntax of terms at different stages.

The \m{interp/coqlib.ml} file collects and builds many examples of
global references to parts of the Coq standard library and it can be
used as an example for reference creation.

\newpage

\section{Starting the system}

The \m{toplevel/coqtop.mli} file contains the following function:

\begin{lstlisting}
val start : unit -> unit
\end{lstlisting}

Running it will start the \m{coqtop} interactive loop. The code for
the loop itself can be found in \m{toplevel/toplevel.ml}. What happens
in the loop is essentially calling \m{raw\_do\_vernac} (from
\m{toplevel/vernac.ml}) with the user input (plus error
handling). This function is in turn the composition of the following
two functions (also from \m{toplevel/vernac.ml}):

\begin{lstlisting}
parse_sentence : Pcoq.Gram.parsable * in_channel option ->
 Util.loc * Vernacexpr.vernac_expr
val eval_expr : ?preserving:bool -> Util.loc * Vernacexpr.vernac_expr -> unit
\end{lstlisting}

\noindent
The \m{eval\_expr} is a (yet another!) wrapper on the following
function (from \m{toplevel/vernacentries.ml}):

\begin{lstlisting}
(** The main interpretation function of vernacular expressions *)
val interp : Vernacexpr.vernac_expr -> unit
\end{lstlisting}

\noindent
By inspecting the source code of the \m{interp} function you should be
able to find your way to the module in which the given Vernacular
command is handled.

\newpage

\section{Dynamic linking}

The \m{toplevel/mltop.ml} seems to be concerned with the dynamic
loading of ML modules. I'm not sure how does this work yet, but this
module is probably used to implement the plugin mechanism.

\section{Pretty printing}

The \m{lib/pp.ml} file defines the basic primitives for pretty
printing. There are also some additional modules in the \m{parsing}
directory for pretty printing different kinds of entities:

\begin{itemize}
\item \m{parsing/pptactic.ml} for tactics
\item \m{parsing/ppvernac.ml} for Vernacular commands:
\end{itemize}

\noindent
\textbf{Example:} a function for pretty printing Vernacular commands:

\begin{code}
let show_pretty_vernac_expr : Vernacular.vernac_expr -> string = 
  fun c -> Pp.string_of_ppcmds (Ppvernac.pr_vernac c)
\end{code}

\noindent
\textbf{Example:} pretty-printing of a raw tactic:

\begin{code}
let env = Global.env () in
let tac_str = Pp.string_of_ppcmds (Pptactic.pr_raw_tactic env tac) in
Printf.eprintf "debug: %s\n%!" tac_str
\end{code}

\section{The proof engine and tactics}

To start the proof editing mode, the following function (from
\m{proofs/proof\_global.ml}) is used:
\begin{lstlisting}
val start_proof : Names.identifier -> 
                  Decl_kinds.goal_kind ->
                  (Environ.env * Term.types) list  ->
                  ?compute_guard:lemma_possible_guards -> 
                  Tacexpr.declaration_hook -> 
                  unit
\end{lstlisting}

One interesting point is that the proof editing loop (the proof
engine) is implemented imperatively: the current state is kept in a
mutable variable (defined in the \m{proofs/proof\_global.ml} file). I
think this is done because many commands can be issued at all times
(e.g. \m{Check}), so it makes sense to have all of the commands
handled by a single interpretation function; proof-editing commands
are then simply delegated to proof-specific modules. To check if the
proof-editing mode is on, we can call the following function (from
\m{proofs/pfedit.ml}):
\begin{lstlisting}
val refining : unit -> bool
\end{lstlisting}
\textbf{Warning:} tactics invoked by the user are encoded as
the \m{VernacSolve} command constructor (the more intuitive
\m{VernacProof} is used for something different. \TODO{For what
  exactly??})

The abstract syntax tree of tactics can be found in
\m{proofs/tacexpr.ml}. There are two variants: \m{raw\_tactic\_expr}
and \m{glob\_tactic\_expr}. The first one is the direct output of the
parser, the other one is used in the interpretation. To translate from
the \m{raw} version to the \m{glob} version either use the top-level
\m{glob\_tactic} or one of the more fine-grained \m{intern\_*} functions
(both can be found in \m{tactics/tacinterp.ml}).

To interpret (execute) tactic expressions check the
\m{tactics/tacinterp.ml} file. The interesting functions are:
\begin{lstlisting}
val interp : raw_tactic_expr -> tactic
val eval_tactic : glob_tactic_expr -> tactic
\end{lstlisting}
The interpretation of atomic tactics is split between various files in
the \m{tactics} directory; some (e.g. \m{intro}, \m{apply}) can be
found in the \m{tactics/hiddentac.ml} file.

Named tactic registration is done by the function (again, from
\m{tactics/tacinterp.ml}):
\begin{lstlisting}
val add_tacdef :
  Vernacexpr.locality_flag -> bool ->
  (Libnames.reference * bool * raw_tactic_expr) list -> unit
\end{lstlisting}

The \m{parsing/pptactic.ml} file contains functions for
pretty-printing (both raw and globalized) tactics:

\begin{lstlisting}
val pr_raw_tactic    : env -> raw_tactic_expr -> std_ppcmds
val pr_glob_tactic : env -> glob_tactic_expr -> std_ppcmds
\end{lstlisting}

\chapter{Source code organization}

\section{Toplevel directories}

The source code of Coq 8.4pl3 is organized into the following
subdirectories:
%% The toplevel directory for the Coq 8.4pl3 version contains the
%% following subdirectories:
\begin{lstlisting}
bin checker config dev doc ide interp kernel lib library man parsing
plugins pretyping proofs scripts states tactics test-suite theories
tools toplevel
\end{lstlisting}

\noindent
The toplevel structure can be summarized as follows: \\

\begin{tabular}{|r|c|}
\hline
bin & target dir for executable files (e.g. \m{coqc}, \m{coqtop}) \\ \hline
checker & \\ \hline
config & \\ \hline
dev & \\ \hline
doc & \\ \hline
ide & \\ \hline
interp & translations between global, local and internal term representations 
\\ \hline
kernel & \\ \hline
lib & various helper modules (e.g. pretty-printing, locations, etc)
\\ \hline
library & \\ \hline
man & \\ \hline
parsing & parsing and pretty-printing \\ \hline
plugins & \\ \hline
pretyping & \\ \hline
proofs & the proof engine, basic tacticals \\ \hline
scripts & \\ \hline
states & \\ \hline
tactics & atomic tactics and ltac expressions\\ \hline
test-suite & \\ \hline
theories & \\ \hline
tools & \\ \hline
toplevel & basic interaction with the user, vernacular command handling
\\ \hline
\end{tabular}

We now proceed to discuss the contents of some of the directories in
more detail.

\subsection{bin}
\subsection{checker}
\subsection{config}
\subsection{dev}
\subsection{doc}
\subsection{ide}
\subsection{interp}
\subsection{kernel}
\subsection{lib}
\subsection{library}
\subsection{man}
\subsection{parsing}
\subsection{plugins}
\subsection{pretyping}
\subsection{proofs}
\subsection{scripts}
\subsection{states}
\subsection{tactics}
\subsection{test-suite}
\subsection{theories}
\subsection{tools}
\subsection{toplevel}

\chapter{Source code -- module by module}

\section{Kernel}

\subsection{Names}

The \m{kernel/names.ml} file is concerned with the variety of names
that are used throughout Coq's source code. Many of the name types are
kept abstract (e.g. \m{identifier}, \m{label}, etc) and only
conversion functions are exported (e.g. \m{string\_of\_id}).

The main name types defined in the file are:

\begin{code}
type identifier
type name = Name of identifier | Anonymous
type variable = identifier

type module_ident = identifier
type dir_path       (* Directory paths = section names paths *)
type label          (* Names of structure elements *)
type mod_bound_id   (* Unique names for bound modules *)
type module_path =
  | MPfile of dir_path
  | MPbound of mod_bound_id
  | MPdot of module_path * label

type kernel_name     (* The absolute names of objects seen by kernel *)

type constant
type mutual_inductive
type inductive = mutual_inductive * int
type constructor = inductive * int
\end{code}

\noindent
For more details and additional comments check \m{kernel/names.mli}. \\

\noindent
\textbf{Bonus}: a function for building \textbf{constants} of form
\m{Equations.DepElim}.\emph{name}:

\begin{code}
let build_lemma_constant : string -> constant =
  fun lemma_name ->
    let lemma_label = Names.mk_label lemma_name in
    let dir_path = Names.empty_dirpath in
    let inner_dir_path = List.map Names.id_of_string [ "DepElim" ; "Equations" ] in
    let module_path = Names.MPfile (Names.make_dirpath inner_dir_path ) in
    Names.make_con module_path dir_path lemma_label
\end{code}
Note how we have to reverse the list when constructing \m{inner\_dir\_path}.

\subsection{Term}

The \m{kernel/term.ml} file is concerned with the definition of the
basic constructions of the CIC. The two most important types seem to
be \m{sorts} and \m{constr}. The latter is the type of constructions
and is abstract, but many constructors (in the form \m{mkABC},
e.g. \m{mkSet}), (partial) destructors (e.g. \m{destApp}) and helper
functions are provided (including substitution). 

For cases where one would like to pattern match on constr, a
\emph{view} \m{kind\_of\_term} is provided along with the
destructuring function (also called \m{kind\_of\_term}).

The summary of most important types is as follows:

\begin{code}
type contents = Pos | Null
type sorts =
  | Prop of contents       (** Prop and Set *)
  | Type of Univ.universe  (** Type *)
type sorts_family = InProp | InSet | InType

type existential_key = int
type metavariable = int

type case_stype = ...
type case_printing = ...
type case_info = ...

type constr
type types = constr   (* 'type' is an ML keyword *)

type existential = existential_key * constr array
type rec_declaration = name array * types array * constr array
type fixpoint = (int array * int) * rec_declaration
type cofixpoint = int * rec_declaration

type ('constr, 'types) kind_of_term =
  | Rel       of int
  | Var       of identifier
  | Meta      of metavariable
  | Evar      of 'constr pexistential
  | Sort      of sorts
  | Cast      of 'constr * cast_kind * 'types
  | Prod      of name * 'types * 'types
  | Lambda    of name * 'types * 'constr
  | LetIn     of name * 'constr * 'types * 'constr
  | App       of 'constr * 'constr array
  | Const     of constant
  | Ind       of inductive
  | Construct of constructor
  | Case      of case_info * 'constr * 'constr * 'constr array
  | Fix       of ('constr, 'types) pfixpoint
  | CoFix     of ('constr, 'types) pcofixpoint

type ('constr, 'types) kind_of_type =
  | SortType   of sorts
  | CastType   of 'types * 'types
  | ProdType   of name * 'types * 'types
  | LetInType  of name * 'constr * 'types * 'types
  | AtomicType of 'constr * 'constr array

type named_declaration = identifier * constr option * types
type rel_declaration = name * constr option * types

type arity = rel_context * sorts
type values
\end{code}

\newpage

\section{Library}

\subsection{Libnames}

The \m{library/libnames.ml} file defines a few notions of qualified
names and provides functions for manipulating them.

The most important types seem to be:

\begin{code}
type global_reference =
  | VarRef of variable
  | ConstRef of constant
  | IndRef of inductive
  | ConstructRef of constructor

type syndef_name = kernel_name
type extended_global_reference =
  | TrueGlobal of global_reference
  | SynDef of syndef_name

type full_path
type qualid
type object_name = full_path * kernel_name
type object_prefix = dir_path * (module_path * dir_path)

type global_dir_reference = ...

type reference =
  | Qualid of qualid located
  | Ident of identifier located
\end{code}

\newpage

\section{Proofs}

\subsection{Refiner}

The file \m{proofs/refiner.ml} contains a large list of primitive
tacticals that can be composed to create larger tactics. During the
interpretation of the Ltac tactical language (in file
\m{tactics/tacinterp.ml}) the refiner tacticals are used once all of
the arguments have been calculated, so in a way they are the low-level
goal-manipulators.

A few example tacticals:

\begin{code}
val tclIDTAC  : tactic
val tclREPEAT : tactic -> tactic
val tclTRY    : tactic -> tactic
val tclFAIL   : int -> Pp.std_ppcmds -> tactic
\end{code}

\subsection{Tacexpr}

The \m{proofs/tacexpr.ml} file contains the definitions of tactic
(Ltac) expressions. The inductive definitions have many
arguments. This is done so that there can be two variants of tactic
expressions, raw expressions from parsing:
\begin{lstlisting}
type raw_tactic_expr =
    (constr_expr,
     constr_pattern_expr,
     reference or_by_notation,
     reference or_by_notation,
     reference,
     identifier located or_metaid,
     raw_tactic_expr,
     rlevel) gen_tactic_expr
\end{lstlisting}
and the expressions after some initial checks:
\begin{lstlisting}
type glob_tactic_expr =
    (glob_constr_and_expr,
     glob_constr_and_expr * constr_pattern,
     evaluable_global_reference and_short_name or_var,
     inductive or_var,
     ltac_constant located or_var,
     identifier located,
     glob_tactic_expr,
     glevel) gen_tactic_expr
\end{lstlisting}

The translation from the first variant to the second one is done in
the \m{tactics/tacinterp.ml} file.

\newpage

\section{Tactics}

\subsection{Tacinterp}

The \m{tactics/tacinterp.ml} file is concerned with the interpretation
of tactic expressions.

Some of the operations defined in the file are for the internalization
of the tactic expressions:

\begin{code}
val intern_pure_tactic : glob_sign -> raw_tactic_expr -> glob_tactic_expr
val intern_constr : glob_sign -> constr_expr -> glob_constr_and_expr

val glob_tactic : raw_tactic_expr -> glob_tactic_expr

val add_tacdef :
  Vernacexpr.locality_flag -> bool ->
  (Libnames.reference * bool * raw_tactic_expr) list -> unit

val interp : raw_tactic_expr -> tactic
val eval_tactic : glob_tactic_expr -> tactic
\end{code}

\noindent
The \m{intern\_} functions seem to be only used locally, inside
\m{glob\_tactic}.

The \m{add\_tacdef} function is called by the Vernacular when the user
registers a new tactic (by \verb|Ltac foo := ...|).

The \m{interp} and \m{eval\_tactic} functions are used to evaluate
Ltac expressions. Which one is used depends on the \m{raw}/\m{glob}
status.

\newpage

\section{Interp}

\subsection{Topconstr}

The \m{interp/topconstr.ml} file defines, among others, the
\m{aconstr} and \m{constr\_expr} term representations. Also, it seems
to deal with notations and their interpretations and
matchings. \TODO{Explain more about aconstr and constr\_expr}.

The most important type definitions are:

\begin{code}
type aconstr =
  (** Part common to [glob_constr] and [cases_pattern] *)
  | ARef of global_reference
  | AVar of identifier
  | AApp of aconstr * aconstr list
  | AList of identifier * identifier * aconstr * aconstr * bool
  (** Part only in [glob_constr] *)
  | ALambda of name * aconstr * aconstr
  | AProd of name * aconstr * aconstr
  | ABinderList of identifier * identifier * aconstr * aconstr
  | ALetIn of name * aconstr * aconstr
  | ACases of case_style * aconstr option *
      (aconstr * (name * (inductive * int * name list) option)) list *
      (cases_pattern list * aconstr) list
  | ALetTuple of name list * (name * aconstr option) * aconstr * aconstr
  | AIf of aconstr * (name * aconstr option) * aconstr * aconstr
  | ARec of fix_kind * identifier array *
      (name * aconstr option * aconstr) list array * aconstr array *
      aconstr array
  | ASort of glob_sort
  | AHole of Evd.hole_kind
  | APatVar of patvar
  | ACast of aconstr * aconstr cast_type

type scope_name = string

type notation_var_instance_type =
  | NtnTypeConstr | NtnTypeConstrList | NtnTypeBinderList
type notation_var_internalization_type =
  | NtnInternTypeConstr | NtnInternTypeBinder | NtnInternTypeIdent
type interpretation =
    (identifier * (subscopes * notation_var_instance_type)) list * aconstr

type notation = string
type explicitation = ...
type binder_kind = ...
type abstraction_kind = ...
type proj_flag = int option
type prim_token = Numeral of Bigint.bigint | String of string

type cases_pattern_expr =
  | CPatAlias of loc * cases_pattern_expr * identifier
  | CPatCstr of loc * reference * cases_pattern_expr list
  | CPatCstrExpl of loc * reference * cases_pattern_expr list
  | CPatAtom of loc * reference option
  | CPatOr of loc * cases_pattern_expr list
  | CPatNotation of loc * notation * cases_pattern_notation_substitution
  | CPatPrim of loc * prim_token
  | CPatRecord of Util.loc * (reference * cases_pattern_expr) list
  | CPatDelimiters of loc * string * cases_pattern_expr

and cases_pattern_notation_substitution =
    cases_pattern_expr list *     (** for constr subterms *)
    cases_pattern_expr list list  (** for recursive notations *)

type constr_expr =
  | CRef of reference
  | CFix of loc * identifier located * fix_expr list
  | CCoFix of loc * identifier located * cofix_expr list
  | CArrow of loc * constr_expr * constr_expr
  | CProdN of loc * (name located list * binder_kind * constr_expr) list * constr_expr
  | CLambdaN of loc * (name located list * binder_kind * constr_expr) list * constr_expr
  | CLetIn of loc * name located * constr_expr * constr_expr
  | CAppExpl of loc * (proj_flag * reference) * constr_expr list
  | CApp of loc * (proj_flag * constr_expr) *
      (constr_expr * explicitation located option) list
  | CRecord of loc * constr_expr option * (reference * constr_expr) list
  | CCases of loc * case_style * constr_expr option *
      (constr_expr * (name located option * constr_expr option)) list *
      (loc * cases_pattern_expr list located list * constr_expr) list
  | CLetTuple of loc * name located list * (name located option * constr_expr option) *
      constr_expr * constr_expr
  | CIf of loc * constr_expr * (name located option * constr_expr option)
      * constr_expr * constr_expr
  | CHole of loc * Evd.hole_kind option
  | CPatVar of loc * (bool * patvar)
  | CEvar of loc * existential_key * constr_expr list option
  | CSort of loc * glob_sort
  | CCast of loc * constr_expr * constr_expr cast_type
  | CNotation of loc * notation * constr_notation_substitution
  | CGeneralization of loc * binding_kind * abstraction_kind option * constr_expr
  | CPrim of loc * prim_token
  | CDelimiters of loc * string * constr_expr

type with_declaration_ast = ...
type module_ast = ...
\end{code}

\subsection{Genarg}

The \m{interp/genarg.ml} module explains the path an entity can take
from parsing to evalution. It also introduces a plethora of phantom
types in order to differentiate between raw, globalized and
internalized (interpreted?) expressions. It also seems to introduce
some form of (safe?) dynamic typing.

\subsection{Constrextern}

The \m{interp/constrextern.ml} file contains a bunch of functions from
the internal syntaxes (\m{glob\_constr}, \m{constr}) to the external
syntax (\m{constr\_expr}) for printing. It also contains a lot of
options that allow us to decide if we e.g. want to print implicit
arguments.

\subsection{Constrintern}

The \m{interp/constrintern.ml} file deals with the translation from
the front abstract syntax (\m{constr\_expr} to the internal syntaxes
(\m{glob\_constr}, \m{constr}).

\subsection{Coqlib}

The \m{interp/coqlib.ml} file contains \textbf{global references} to
the parts of the Coq standard library that are used in Ocaml files.
The implementation seems instructive for handling the manual creation
of global references.

Some of the most useful parts seem to be:

\begin{code}
type message = string
val find_reference : message -> string list -> string -> global_reference
(** For tactics/commands requiring vernacular libraries *)
val check_required_library : string list -> unit

val logic_module : dir_path
val logic_type_module : dir_path

(* For Equality tactics *)
type coq_sigma_data = {
  proj1 : constr;
  proj2 : constr;
  elim  : constr;
  intro : constr;
  typ   : constr }

val build_sigma_set : coq_sigma_data delayed
val build_sigma_type : coq_sigma_data delayed
val build_sigma : coq_sigma_data delayed
\end{code}

\section{Pretyping}

\subsection{Glob\_term}

In file \m{pretyping/glob_term} the \m{glob\_constr} intermediate term
representation is defined. The file contains also definitions of
the patterns that occur inside 'match ... with' expressions.

Most important type definitions are:

\begin{code}
type cases_pattern =
  | PatVar of loc * name
  | PatCstr of loc * constructor * cases_pattern list * name 
type patvar = identifier

type quantified_hypothesis = AnonHyp of int | NamedHyp of identifier
type 'a explicit_bindings = (loc * quantified_hypothesis * 'a) list
type 'a bindings =
  | ImplicitBindings of 'a list
  | ExplicitBindings of 'a explicit_bindings
  | NoBindings
type 'a with_bindings = 'a * 'a bindings

type 'a cast_type =
  | CastConv of cast_kind * 'a
  | CastCoerce (** Cast to a base type (eg, an underlying inductive type) *)

type glob_constr =
  | GRef of (loc * global_reference)
  | GVar of (loc * identifier)
  | GEvar of loc * existential_key * glob_constr list option
  | GPatVar of loc * (bool * patvar) (** Used for patterns only *)
  | GApp of loc * glob_constr * glob_constr list
  | GLambda of loc * name * binding_kind *  glob_constr * glob_constr
  | GProd of loc * name * binding_kind * glob_constr * glob_constr
  | GLetIn of loc * name * glob_constr * glob_constr
  | GCases of loc * case_style * glob_constr option * tomatch_tuples * cases_clauses
      (** [GCases(l,style,r,tur,cc)] = "match 'tur' return 'r' with 'cc'" (in 
	  [MatchStyle]) *)

  | GLetTuple of loc * name list * (name * glob_constr option) *
      glob_constr * glob_constr
  | GIf of loc * glob_constr * (name * glob_constr option) * glob_constr * glob_constr
  | GRec of loc * fix_kind * identifier array * glob_decl list array *
      glob_constr array * glob_constr array
  | GSort of loc * glob_sort
  | GHole of (loc * Evd.hole_kind)
  | GCast of loc * glob_constr * glob_constr cast_type

and glob_decl = ...

type 'a glob_red_flag = ...

type 'a or_var = ArgArg of 'a | ArgVar of identifier located
type occurrences_expr = bool * int or_var list
type 'a with_occurrences = occurrences_expr * 'a

type ('a,'b,'c) red_expr_gen = ...
type ('a,'b,'c) may_eval =
  | ConstrTerm of 'a
  | ConstrEval of ('a,'b,'c) red_expr_gen * 'a
  | ConstrContext of (loc * identifier) * 'a
  | ConstrTypeOf of 'a
\end{code}

\section{Lib}

\subsection{Util}

The \m{lib/util.ml} file defines many kinds of types and values used
throughout the whole source base. It contains utilities for error
reporting, location handling, operations on pairs and triples, various
functions on chars and strings (including unicode support), a bunch of
list and array functionals, some combinators for streams, matrices and
functions. Also there are integer sets and maps and pretty printing
combinators. Finally, there is support for function memoization.

There aren't many types defined in this file:

\begin{code}
type loc = Loc.t
type 'a located = loc * 'a

type 'a delayed = unit -> 'a

type ('a,'b) union = Inl of 'a | Inr of 'b
\end{code}

\chapter{All about names and terms}

\TODO{glob\_term, topconstr, term -- explain which representation is
  used for what!}

According to the comment in \m{pretyping/glob\_term.mli} the
\m{glob\_term} type is an intermediate term representation and it is
used for
\begin{itemize}
\item name resolution
\item implicit argument placeholder insertion
\item notation handling
\end{itemize}
while coercions, inference of implicit arguments and pattern-matching
compilation are \textbf{not} done on this representation.

We have the following identifier-like types:

\begin{code}
(* Kernel.Names *)
type identifier
type name
type variable = identifier
type module_ident
type dir_path
type label
type mod_bound_id
type module_path
type kernel_name
type constant
type evaluable_global_reference
(* Library.Libnames *)
type global_reference
type full_path
type qualid
type object_name
type object_prefix
type global_dir_reference
type reference    
\end{code}

Some of the functions on identifiers are:

\begin{code}
(* Interp.Topconstr *)
coerce_reference_to_id : reference -> identifier
coerce_to_id : constr_expr -> identifier located
coerce_to_name : constr_expr -> name located
\end{code}

\section{Terms (constructions)}

We have the following type definitions:

\begin{code}
(* Pretyping.Glob_term *)
type patvar
type glob_constr
(* Interp.Topconstr *)
type aconstr
type constr_expr
type constr_pattern_expr
(* Interp.Genarg *)
type glob_constr_and_expr = glob_constr * constr_expr option
type open_constr_expr = unit * constr_expr
type open_glob_constr = unit * glob_constr_and_expr

rawwit_constr  : (constr_expr,          rlevel) abstract_argument_type
globwit_constr : (glob_constr_and_expr, glevel) abstract_argument_type
wit_constr     : (constr,               tlevel) abstract_argument_type
(* Kernel.Term *)
type sorts
type constr
type types
type kind_of_term
(* Pretyping.Evd *)
type open_constr = evar_map * constr
\end{code}

We have the following conversion functions:

\begin{code}
(* Interp.Topconstr *)
glob_constr_of_aconstr : loc -> aconstr -> glob_constr
(* Interp.Constrintern *)
intern_constr : evar_map -> env -> constr_expr -> glob_constr
interp_constr : evar_map -> env -> constr_expr -> constr
interp_open_constr : evar_map -> env -> constr_expr -> evar_map * constr
interp_casted_constr : evar_map -> env -> ?impls:internalization_env -> 
                       constr_expr -> types -> constr
(* Interp.Constrextern *)
extern_glob_constr : Idset.t -> glob_constr -> constr_expr
extern_constr : bool -> env -> constr -> constr_expr
extern_reference : loc -> Idset.t -> global_reference -> reference
\end{code}

\section{Terms (tactics)}

There are two types for tactic expressions: \m{raw\_tactic\_expr} and
\m{glob\_tactic\_expr}. We have described them in other sections, see
\TODO{provide back links}.

\part{Tools and techniques}

\chapter{Plugin creation}

\TODO{Reference (discuss?) Matthieu Sozeau's Constructor example from
  github.}

\part{Reference manual}

\chapter{The interfaces}

This chapter could contain the listings of all of the original *.mli
files. On the other hand, we already provide much of that information
in the previous sections, so it might be a waste of space...

\chapter{Index}

This chapter could contain index(es) of all types, modules, values
etc. A similar thing is provided by the documentation extracted from
the .mli files by the ocamldoc tool, so maybe it's not very good idea
to repeat those lists here. However, a list of some sort could still
be useful.

\chapter{References}

This chapter will contain references to the relevant literature.

\end{document}
